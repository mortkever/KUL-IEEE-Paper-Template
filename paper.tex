\documentclass[conference]{IEEEtran}
\IEEEoverridecommandlockouts
% The preceding line is only needed to identify funding in the first footnote. If that is unneeded, please comment it out.
\usepackage{amsmath,amssymb,amsfonts}
\usepackage{algorithmic}
\usepackage{graphicx}
\usepackage{textcomp}
\usepackage{xcolor}
\usepackage[backend=biber, style=ieee, 
citestyle=numeric-comp, maxnames=50, maxcitenames=1]{biblatex}
\addbibresource{bib.bib}
\def\BibTeX{{\rm B\kern-.05em{\sc i\kern-.025em b}\kern-.08em
    T\kern-.1667em\lower.7ex\hbox{E}\kern-.125emX}}
\usepackage[acronym,xindy]{glossaries}
\makenoidxglossaries
\newacronym{2d}{2D}{two-dimensional}
\newacronym{dfi}{DFI}{data-flow integrity}
\newacronym{cfi}{CFI}{control-flow integrity}
\begin{document}

\title{PTAsan: Detecting Unsoundness in Pointer Analysis Results\\
{\footnotesize \textsuperscript{*}The Unharmoniuous Truth of Unsound Pointer Analysis}
}

\author{\IEEEauthorblockN{1\textsuperscript{st} Victor Deforce}
    \IEEEauthorblockA{\textit{dept. name of organization (of Aff.)} \\
        \textit{name of organization (of Aff.)}\\
        City, Country \\
        email address or ORCID}
    \and
    \IEEEauthorblockN{2\textsuperscript{nd} Given Name Surname}
    \IEEEauthorblockA{\textit{dept. name of organization (of Aff.)} \\
        \textit{name of organization (of Aff.)}\\
        City, Country \\
        email address or ORCID}
    \and
    \IEEEauthorblockN{3\textsuperscript{rd} Given Name Surname}
    \IEEEauthorblockA{\textit{dept. name of organization (of Aff.)} \\
        \textit{name of organization (of Aff.)}\\
        City, Country \\
        email address or ORCID}
    \and
    \IEEEauthorblockN{4\textsuperscript{th} Given Name Surname}
    \IEEEauthorblockA{\textit{dept. name of organization (of Aff.)} \\
        \textit{name of organization (of Aff.)}\\
        City, Country \\
        email address or ORCID}
    \and
    \IEEEauthorblockN{5\textsuperscript{th} Given Name Surname}
    \IEEEauthorblockA{\textit{dept. name of organization (of Aff.)} \\
        \textit{name of organization (of Aff.)}\\
        City, Country \\
        email address or ORCID}
    \and
    \IEEEauthorblockN{6\textsuperscript{th} Given Name Surname}
    \IEEEauthorblockA{\textit{dept. name of organization (of Aff.)} \\
        \textit{name of organization (of Aff.)}\\
        City, Country \\
        email address or ORCID}
}

\maketitle

\begin{abstract}
    Een heel aantal toepassingen maken gebruik van pointer analysis: compiler optimalization, Data Flow Integrity (DFI), control flow integrity, compartmentalization, etc. 
Hoewel het algemeen geweten is dat deze analyses een overschatting zijn, staat men er zelden bij stil dat de analyse ook points-to sets kan ónderschatten. Deze onderschatting 
van points-to sets door 
deze tools is nefast voor toepassingen zoals DFI en compartmentilization. Deze toepassingen steunen
op de overschatting van de set van adressen, die een lees- of schrijfinstructie kan benaderen, met
crashes op willekeurige momenten tot gevolg als dit niet zo is.
Er is geen punt aan veiligheidsmechanismes zoals DFI, als ze random crashes veroorzaken doordat de 
points-to sets te beperkend zijn.
Er is dus een
duidelijke kloof tussen ontwikkelaars van deze pointer analysis tools, die de veronderstellingen en
limieten in hun implementatie kennen, en de ontwikkelaars die deze tools gebruiken in hun
toepassingen. 

In deze thesis hebben we de PTAsan, Pointer Analysis sanitizer, ontwikkeld om de
aanwezigheid van unsoundness in veel gebruikte pointer analyses empirisch te meten. De PTAsan
geeft ons inzicht in de oorzaak van de unsoundness via uitgebreide logging. We hebben de PTAsan
geïmplementeerd als een DFI-like controlesysteem en we instrumenteren testcode door middel van
LLVM. We hebben o.a. de Andersen pointer analysis van SVF getest op de SPEC CPU2006
benchmarking suite en vastgesteld dat tot 2\% van de points-to sets unsound zijn. Aangezien grotere
programma's zoals GCC bestaan uit meer dan honderdduizend points-to sets, zorgt dit voor
onpraktisch veel crashes in bijvoorbeeld DFI. We hebben manueel enkele van deze bronnen van
unsoundness opgespoord en gerapporteerd aan de ontwikkelaars van SVF.
\end{abstract}

\begin{IEEEkeywords}
    Pointer analysis, data-flow integrity, SVF, Soundness, soundiness
\end{IEEEkeywords}

\section{Introductie}
Dezelfde vrijheid die pointers geeft in C en C++ zorgt er ook voor dat programma's geschreven in deze taal gevoelig zijn voor aanvallen met geheugen exploits. 
Doorheen de jaren zijn tal van oplossingen uitgewerkt waaronder \gls{dfi}\cite{castro_securing_nodate} en WIT\cite{akritidis_preventing_2008}. \gls{dfi} bepaald 
statisch, met een pointer analyse, voor elke lees en schrijf instructie de (points-to) set van objecten die deze instructie zal kunnen benaderen onder normale 
omstandigheden. Terwijl het programma loopt, dwingt \gls{dfi} deze sets af door het programma te laten crashen indien de instructie benadert dat niet in de set zit. 
Als bijvoorbeeld een aanvaller de normale verloop van het programma probeert te wijzigen door return adressen te overschrijven, dan zal \gls{dfi} dit detecteren en 
de aanval afwenden.

Om dus een goede werking te garanderen mag de statische analyse geen onderschatting maken van de objecten in de points-to set. Als analyse objecten mist in de set kan 
dit leiden tot spontane crashes in het programma wanneer er geen aanval gaande is. Hoewel dat ontwikkelaars die deze analyses gebruiken zich over het algemeen bewust 
van de overschattingen die een pointer analyse maakt, staat men er zelden bij stil dat de analyse ook sets kan onderschatten.
Pointer analyses worden in vele toepassingen gebruikt. We delen ze daarom hier op in twee groepen. Voor de eerste groep gebruiken we de term \textbf{offensieve pointer analyse}.
Deze analyses worden voornamelijk gebruikt voor allerhande bug-detectie tools: memory leaks, ongeïnitialiseerde variabelen, 
etc.\cite{sui_svf_nodate}\cite{cherem_practical_nodate}\cite{ye_accelerating_nodate}. Daarnaast zijn er, wat Smaragdakis\cite{smaragdakis_defensive_2018} noemt 
\textbf{defensieve pointer analysis}. Deze worden gebruikt o.a.\ door \gls{dfi}\cite{castro_securing_nodate}\cite{diez-franco_optimized_2024}, 
\gls{cfi}\cite{kasten_integrating_2024}\cite{li_finding_2020} en een aantal compiler optimalisaties\cite{hind_interprocedural_1999}. 

We maken het onderscheid tussen deze twee groepen aangezien hun verwachtingen loodrecht op elkaar staan. Voor bijvoorbeeld het detecteren van memory leaks heeft men 
een analyse nodig die snel en precies is\cite{cherem_practical_nodate}\cite{fan_smoke_2019}. Want de analyse draait mogelijks in de IDE en men ziet liefst zo weinig 
mogelijk foute rapporteringen. Het is minder relevant dat de detector enkele leaks mist, aangezien de resultaten in vergelijking met andere implementaties worden gepubliceerd.
Een analyse die hier aan voldoet noemen we een offensieve analyse.

Toepassingen die een defensieve pointer analyses gebruiken zoals \gls{dfi}, verwachten ook een zekere precisie maar steunen op de garantie dat de sets niet onderschat zijn en dus 
sound zijn. Zoals eerder vermeld leidt een onderschatting van de sets tot een analyse die niet functioneert. We gebruiken dan ook de term defensief om te duiden welke garanties deze 
analyses zouden moeten stellen. Voor zover ons bekend, bestaan er slechts drie analyses die beweren defensief te zijn.\cite{smaragdakis_defensive_2018}\cite{cai_unleashing_nodate}\cite{lu_practical_2023}

Hoewel het aantal defensieve analyses beperkt is en de ontwikkelaars zich wel bewust zijn van de beperkingen in hun implementaties\cite{vojnar_phasar_2019}, beseffen weinig 
developers die de analyses gebruiken dat deze sound zijn. Vele zijn dus overhead en compatibiliteits problemen als de voornaamste obstakels\cite{mathiasen_fine-grained_2021}\cite{bellec_rt-dfi_2022}\cite{diez-franco_optimized_2024}\cite{feng_toward_2022} 
voor \gls{dfi} buiten de onderzoekswereld gebruikt kan worden. Echter zonder sound pointer analyse, zal \gls{dfi} voor random crashes blijven zorgen en dus niet ingezet worden.

Om voor unsoundness te corrigeren voegen sommigen developers van toepassingen een profiling phase toe aan hun implementatie\cite{jin_annotating_2022}\cite{kirth_pkru-safe_2022}.
In deze fase voegen gemiste objecten toe aan de sets tijdens test runs om zo crashes te vermijden tijdens de benchmarks. Dynamische analyses hebben echter een coverage probleem 
want met de test cases waarmee het programma is getest, kan men zelden het volledige gedrag van het programma opbouwen. Het programma is dus sound voor de testcases maar 
waarschijnlijk niet voor andere inputs.

In het soundiness manifesto\cite{livshits_defense_2015} doet een groep van pointer analysis onderzoekers een oproep voor meer 
transparantie. Ze stellen dat soundness een afweging vereist met precisie en schaalbaarheid. Bepaald gedrag in een programma, 
zoals pointer arithmetic, maken het moeilijk om precies de points-to sets van een pointer te bepalen. Men stelt dus de term 
soundiness voor, het sound behandelen van de meeste language features maar het bewust onderschatten van sets voor zekere 
ver onderzochte feature set. 

Om de verregaande aanwezigheid van unsoundness te empirisch te bewijzen en te quantificeren hebben we PTAsan ontwikkelt, pointer analysis 
sanitizer. Onze implementatie laat ons toe alle fouten in de points-to set voor een gegeven test programma te loggen en de oorzaak terug 
te volgen tot aan zijn bron. 

\section{Achtergrond}

\subsection{Betekenis van soundness}
In het Engels kunnen de termen sound of soundness gebruikt worden om aan te geven dat iets correct werkt. Afhankelijk van de verwachtingen die men stelt kan hetzelfde dus tegelijk sound en unsound zijn. Zoals in de inleiding al aangegeven kan er een onderscheid in points-to analyses gemaakt worden tussen defensieve en offensieve. Afhankelijk van de categorie kan een resultaat hier dus ook sound of unsound zijn.

Voor offensieve analyses is het doel om precise sets af te leveren die dus een zo klein mogelijke overschatting maken. Als de set meer objecten bevat dan dat werkelijk mogelijk is, duiden mogelijk toepassingen zoals bug-detectie fouten aan waar er geen zijn. Voor de eindgebruiker is dit niet gewenst. Een analyse die teveel overschatting doet in de sets kan men dus als unsound beschouwen.

Voor defensieve analyses is het van belang dat de set exhaustief is. Voor bijvoorbeeld \gls{dfi} zorgt een overschatting dat de veiligheidsgaranties minder sterk zijn, een onderschatting maakt echter het veiligheidsmechanisme rechtuit onbruikbaar. Een defensieve analyse is dus enkel sound als deze geen onderschatte sets als resultaat levert. Dit is al duidelijk in sterk contrast met de offensieve analyses. 

Verder kan de nuance dieper worden doorgetrokken op vlak defensieve analyses. C en C++ laten de programmeur veel vrijheid. Hoe defensief een analyse is kan dus afgemeten worden aan de hand van de bizarre, door programmeurs bedachte, constructies die nog door de analyse sound worden afgehandeld. 

Men zou bijvoorbeeld typisch een access van één object via de pointer van een ander object beschouwen als een memory error. Echter sommige programmeurs passen volgende techniek toe. Ze berekenen de offset tussen de adressen van twee objecten en slaan deze op. Op een later moment gebruiken ze deze offset en de pointer van één van de twee objecten om het andere object te benaderen. Dit noemt men relative pointers. 

Een ander fenomeen waar men op de site van het soundiness manifesto\cite{noauthor_soundiness_nodate} voor bewustzijn voor pleit, is mogelijke buffer overflows. Naar onze mening is dit buiten het correct gedrag van het programma en hoort een analyse hier dus geen rekening mee te nemen. Het sound verwerken van dit fenomeen zou echter de analyse een meer sound resultaat doen geven in relatie tot de reële werking van het programma. 

\subsection{Cascades van unsoundness}

De bron van unsoundness kan ook unsoundness zijn. In deze zin heeft kan een unsound set een cascade effect hebben op andere sets. Dit is het gemakkelijkst te zien bij sets van struct pointers en function pointers. Zo kan een struct met een function pointer als field ontbreken uit de set van een pointer en dan zal de functie ook ontbreken in de sets van de pointers die gelijk worden gesteld aan die function pointer via de struct pointer. Andersom kan een function pointer ook ontbreken in de set van de operand van een call. Logischerwijs zitten de elementen uit de sets van de meegegeven argumenten niet in de sets van de parameters van die functie. 

Aan deze gevallen kunnen we zien dat het traceren van de bron van unsoundness niet alleen neerkomt op het vinden van de oorzaak 
Aan deze gevallen is te zien dat een zinvolle analyse niet alleen ons helpt de bron van unsoundness te vinden maar dat deze ook de eerste oorzaak van unsoundness vindt. Unsound accesses na de eerste unsound access kunnen ook de oorzaak zijn van problemen in de analyse maar enkel voor de eerste is gegarandeerd dat de oorzaak van de unsoundness niet bij andere unsoundness ligt.


\section{Implementatie}
In deze paper leggen de werking uit van PTAsan, de points-to analysis sanitizer. Met PTAsan hebben we als doel het aantal fouten in points-to analysis tools op te meten en voldoende informatie te verzamelen om de oorsprong van deze fouten op te sporen. We doen dit door middel van een door \gls{dfi} en \gls{cfi} geïnspireerde implementatie.

\subsection{PTAsan DFI}
Zoals bij \gls{dfi} bestaat PTAsan uit een compiletime component en een runtime component. At compiletime laten we de te testen points-to analysis de points-to sets opstellen voor de test case. Op basis van deze sets instrumenteren we dan zelf verder het test programma. 

Om elementen uit de sets duidelijk te linken met hun runtime equivalent geven we elk element een uniek kleur. We weten namelijk niet at compiletime waar een specifiek object gealloceerd zal worden en kunnen dus pas at runtime een kleur toekennen aan een specifiek stuk geheugen. Aangezien elk element in een set slaat op een unieke allocation site, niet een uniek object, kunnen we achter elke allocation site een coloring functie invoegen. Deze kleuren zijn 16-bit integers en worden deterministisch toegekend om het traceren van unsoundness te vereenvoudigen.

Deze manier van instrumenteren is geschikt voor statische en dynamische allocaties, echter voor globals die niet worden gealloceerd hebben we een andere aanpak nodig. Voor deze injecteren we een volledig nieuwe functie in de test code. In deze code roepen we voor elke global in het programma de coloring functie op. 

Naast het kleuren van alle allocatie sites, controleren we ook de kleur van een object bij een memory access. We voegen hiervoor ook een function call toe. Aan de color check functie geven we het adres mee van het object en de set van kleuren mee die overeen komen met points-to set van de pointer operand van de access. Verder geven we ook een uniek id mee om unsound accesses te kunnen linken aan de instructie. In de functie verifiëren we of de kleur van het object in de set zit. Als dit het geval is keert de controle terug naar het programma, als dit niet het geval is loggen we de unsound access door middel van de kleur van het object en de id van de instructie.

%Functies uit dynamisch gelinkte libraries kunnen we niet instrumenteren. Hoewel we wel dynamische allocaties instrumenteren, die ook binnen deze categorie vallen, weten we alleen 
Om deze mechanismen te ondersteunen implementeren een aantal functies en datastructuren door middel van een library. De voornaamste data structuur is de color table. Deze bevat voor elke blok van 16 bytes een color. Deze granulariteit is in lijn met minimum grootte van een heap allocated object. Verder definiëren ook een initialisatie functie met constructor attribuut die de color table alloceert alsook de global coloring functie aanroept. Door het constructor attribuut wordt deze functie uitgevoerd voor de main van het test programma. Dit garandeert dat alle setup gebeurd is voor de eerste instructie van het programma zelf wordt uitgevoerd.

Daarnaast bevat onze library ook code om te differentiëren tussen kleurloze objecten. Alleen allocation sites die element zijn van een set krijgen een kleur. Als de points-to analysis dus een bepaalde allocation site niet herkent dan kleuren we deze niet. Dit is voornamelijk een risico bij externe library functies glibc. Deze libraries worden dynamisch gelinkt dus at compiletime is de sourcecode niet beschikbaar. Om dit op te lossen houdt *** (naar background?) een analyse een karakterisatie bij van het pointer gedrag in de deze functie. Echter als deze karakterisering fout is opgesteld, dan kan dit tot unsoundness leiden. Bijvoorbeeld malloc geeft een pointer terug naar nieuw gealloceerd geheugen. Met een foute karakterisering of zelfs met een gebrek aan, ziet de pointer analyse elke malloc niet als een allocation site. 

Via allocator interpostion kunnen we het onderscheid maken tussen allocaties door externe libraries (dynamic allocations) en door andere methoden. Aangezien statisch gelinkte functies eerst gebruikt worden, kunnen we zelf functies definiëren met de naam malloc, calloc, etc. In deze functie kunnen we, als de oorspronkelijke call extern was, een deel van de stacktrace registreren. Het bestaan van deze trace bij een latere unsound access toont dat het object extern gealloceerd was. Als de callchain tussen de eerste externe call en de allocatie niet te groot is kan uit de trace ook worden afgeleid welke functie verkeerd is gekarakteriseerd door de points-to analysis. In de huidige implementatie is deze slechts 1 call diep.

\subsection{PTAsan CFI}
Aangezien unsoundness een cascade effect kan hebben op de resultaten en control flow geen aspect is dat getest wordt onder de \gls{dfi} kant van PTAsan, hebben we een \gls{cfi} kant aan PTAsan toegevoegd. Deze detecteert fouten in het sets van function pointers. De combinatie van beide kanten garandeert ons dat we steeds het eerste soundness issue detecteren.

We implemetneren de \gls{cfi} analoog aan de \gls{dfi} kant.
\section{Resultaten} 



\section{Conclusie}

\subsection{Maintaining the Integrity of the Specifications}

The IEEEtran class file is used to format your paper and style the text. All margins,
column widths, line spaces, and text fonts are prescribed; please do not
alter them. You may note peculiarities. For example, the head margin
measures proportionately more than is customary. This measurement
and others are deliberate, using specifications that anticipate your paper
as one part of the entire proceedings, and not as an independent document.
Please do not revise any of the current designations.

We test our abbreviation here. \gls{2d} technologies are awesome! The world should be more \gls{2d} in our opinion.
\section{Prepare Your Paper Before Styling}
Before you begin to format your paper, first write and save the content as a
separate text file. Complete all content and organizational editing before
formatting. Please note sections \ref{AA}--\ref{SCM} below for more information on
proofreading, spelling and grammar. Here is a \bf{Test textcite}! From \textcite{b1}

Keep your text and graphic files separate until after the text has been
formatted and styled. Do not number text heads---{\LaTeX} will do that
for you.

\subsection{Abbreviations and Acronyms}\label{AA}
Define abbreviations and acronyms the first time they are used in the text,
even after they have been defined in the abstract. Abbreviations such as
IEEE, SI, MKS, CGS, ac, dc, and rms do not have to be defined. Do not use
abbreviations in the title or heads unless they are unavoidable.

\subsection{Units}
\begin{itemize}
    \item Use either SI (MKS) or CGS as primary units. (SI units are encouraged.) English units may be used as secondary units (in parentheses). An exception would be the use of English units as identifiers in trade, such as ``3.5-inch disk drive''.
    \item Avoid combining SI and CGS units, such as current in amperes and magnetic field in oersteds. This often leads to confusion because equations do not balance dimensionally. If you must use mixed units, clearly state the units for each quantity that you use in an equation.
    \item Do not mix complete spellings and abbreviations of units: ``Wb/m\textsuperscript{2}'' or ``webers per square meter'', not ``webers/m\textsuperscript{2}''. Spell out units when they appear in text: ``. . . a few henries'', not ``. . . a few H''.
    \item Use a zero before decimal points: ``0.25'', not ``.25''. Use ``cm\textsuperscript{3}'', not ``cc''.)
\end{itemize}

\subsection{Equations}
Number equations consecutively. To make your
equations more compact, you may use the solidus (~/~), the exp function, or
appropriate exponents. Italicize Roman symbols for quantities and variables,
but not Greek symbols. Use a long dash rather than a hyphen for a minus
sign. Punctuate equations with commas or periods when they are part of a
sentence, as in:
\begin{equation}
    a+b=\gamma\label{eq}
\end{equation}

Be sure that the
symbols in your equation have been defined before or immediately following
the equation. Use ``\eqref{eq}'', not ``Eq.~\eqref{eq}'' or ``equation \eqref{eq}'', except at
the beginning of a sentence: ``Equation \eqref{eq} is . . .''

\subsection{\LaTeX-Specific Advice}

Please use ``soft'' (e.g., \verb|\eqref{Eq}|) cross references instead
of ``hard'' references (e.g., \verb|(1)|). That will make it possible
to combine sections, add equations, or change the order of figures or
citations without having to go through the file line by line.

Please don't use the \verb|{eqnarray}| equation environment. Use
\verb|{align}| or \verb|{IEEEeqnarray}| instead. The \verb|{eqnarray}|
environment leaves unsightly spaces around relation symbols.

Please note that the \verb|{subequations}| environment in {\LaTeX}
will increment the main equation counter even when there are no
equation numbers displayed. If you forget that, you might write an
article in which the equation numbers skip from (17) to (20), causing
the copy editors to wonder if you've discovered a new method of
counting.

    {\BibTeX} does not work by magic. It doesn't get the bibliographic
data from thin air but from .bib files. If you use {\BibTeX} to produce a
bibliography you must send the .bib files.

    {\LaTeX} can't read your mind. If you assign the same label to a
subsubsection and a table, you might find that Table I has been cross
referenced as Table IV-B3.

{\LaTeX} does not have precognitive abilities. If you put a
\verb|\label| command before the command that updates the counter it's
supposed to be using, the label will pick up the last counter to be
cross referenced instead. In particular, a \verb|\label| command
should not go before the caption of a figure or a table.

Do not use \verb|\nonumber| inside the \verb|{array}| environment. It
will not stop equation numbers inside \verb|{array}| (there won't be
any anyway) and it might stop a wanted equation number in the
surrounding equation.

\subsection{Some Common Mistakes}\label{SCM}
\begin{itemize}
    \item The word ``data'' is plural, not singular.
    \item The subscript for the permeability of vacuum $\mu_{0}$, and other common scientific constants, is zero with subscript formatting, not a lowercase letter ``o''.
    \item In American English, commas, semicolons, periods, question and exclamation marks are located within quotation marks only when a complete thought or name is cited, such as a title or full quotation. When quotation marks are used, instead of a bold or italic typeface, to highlight a word or phrase, punctuation should appear outside of the quotation marks. A parenthetical phrase or statement at the end of a sentence is punctuated outside of the closing parenthesis (like this). (A parenthetical sentence is punctuated within the parentheses.)
    \item A graph within a graph is an ``inset'', not an ``insert''. The word alternatively is preferred to the word ``alternately'' (unless you really mean something that alternates).
    \item Do not use the word ``essentially'' to mean ``approximately'' or ``effectively''.
    \item In your paper title, if the words ``that uses'' can accurately replace the word ``using'', capitalize the ``u''; if not, keep using lower-cased.
    \item Be aware of the different meanings of the homophones ``affect'' and ``effect'', ``complement'' and ``compliment'', ``discreet'' and ``discrete'', ``principal'' and ``principle''.
    \item Do not confuse ``imply'' and ``infer''.
    \item The prefix ``non'' is not a word; it should be joined to the word it modifies, usually without a hyphen.
    \item There is no period after the ``et'' in the Latin abbreviation ``et al.''.
    \item The abbreviation ``i.e.'' means ``that is'', and the abbreviation ``e.g.'' means ``for example''.
\end{itemize}
An excellent style manual for science writers is \cite{b7}.

\subsection{Authors and Affiliations}
\textbf{The class file is designed for, but not limited to, six authors.} A
minimum of one author is required for all conference articles. Author names
should be listed starting from left to right and then moving down to the
next line. This is the author sequence that will be used in future citations
and by indexing services. Names should not be listed in columns nor group by
affiliation. Please keep your affiliations as succinct as possible (for
example, do not differentiate among departments of the same organization).

\subsection{Identify the Headings}
Headings, or heads, are organizational devices that guide the reader through
your paper. There are two types: component heads and text heads.

Component heads identify the different components of your paper and are not
topically subordinate to each other. Examples include Acknowledgments and
References and, for these, the correct style to use is ``Heading 5''. Use
``figure caption'' for your Figure captions, and ``table head'' for your
table title. Run-in heads, such as ``Abstract'', will require you to apply a
style (in this case, italic) in addition to the style provided by the drop
down menu to differentiate the head from the text.

Text heads organize the topics on a relational, hierarchical basis. For
example, the paper title is the primary text head because all subsequent
material relates and elaborates on this one topic. If there are two or more
sub-topics, the next level head (uppercase Roman numerals) should be used
and, conversely, if there are not at least two sub-topics, then no subheads
should be introduced.

\subsection{Figures and Tables}
\paragraph{Positioning Figures and Tables} Place figures and tables at the top and
bottom of columns. Avoid placing them in the middle of columns. Large
figures and tables may span across both columns. Figure captions should be
below the figures; table heads should appear above the tables. Insert
figures and tables after they are cited in the text. Use the abbreviation
``Fig.~\ref{fig}'', even at the beginning of a sentence.

\begin{table}[htbp]
    \caption{Table Type Styles}
    \begin{center}
        \begin{tabular}{|c|c|c|c|}
            \hline
            \textbf{Table} & \multicolumn{3}{|c|}{\textbf{Table Column Head}}                                                         \\
            \cline{2-4}
            \textbf{Head}  & \textbf{\textit{Table column subhead}}           & \textbf{\textit{Subhead}} & \textbf{\textit{Subhead}} \\
            \hline
            copy           & More table copy$^{\mathrm{a}}$                   &                           &                           \\
            \hline
            \multicolumn{4}{l}{$^{\mathrm{a}}$Sample of a Table footnote.}
        \end{tabular}
        \label{tab1}
    \end{center}
\end{table}

\begin{figure}[htbp]
    \centerline{\includegraphics{fig1.png}}
    \caption{Example of a figure caption.}
    \label{fig}
\end{figure}

Figure Labels: Use 8 point Times New Roman for Figure labels. Use words
rather than symbols or abbreviations when writing Figure axis labels to
avoid confusing the reader. As an example, write the quantity
``Magnetization'', or ``Magnetization, M'', not just ``M''. If including
units in the label, present them within parentheses. Do not label axes only
with units. In the example, write ``Magnetization (A/m)'' or ``Magnetization
\{A[m(1)]\}'', not just ``A/m''. Do not label axes with a ratio of
quantities and units. For example, write ``Temperature (K)'', not
``Temperature/K''.

\section*{Acknowledgment}

The preferred spelling of the word ``acknowledgment'' in America is without
an ``e'' after the ``g''. Avoid the stilted expression ``one of us (R. B.
G.) thanks $\ldots$''. Instead, try ``R. B. G. thanks$\ldots$''. Put sponsor
acknowledgments in the unnumbered footnote on the first page.

\section*{References}

Please number citations consecutively within brackets \cite{b1}. The
sentence punctuation follows the bracket \cite{b2}. Refer simply to the reference
number, as in \cite{b3}---do not use ``Ref. \cite{b3}'' or ``reference \cite{b3}'' except at
the beginning of a sentence: ``Reference \cite{b3} was the first $\ldots$''

Number footnotes separately in superscripts. Place the actual footnote at
the bottom of the column in which it was cited. Do not put footnotes in the
abstract or reference list. Use letters for table footnotes.

Unless there are six authors or more give all authors' names; do not use
``et al.''. Papers that have not been published, even if they have been
submitted for publication, should be cited as ``unpublished'' \cite{b4}. Papers
that have been accepted for publication should be cited as ``in press'' \cite{b5}.
Capitalize only the first word in a paper title, except for proper nouns and
element symbols.

For papers published in translation journals, please give the English
citation first, followed by the original foreign-language citation \cite{b6}.
\printbibliography

\vspace{12pt}
\color{red}
IEEE conference templates contain guidance text for composing and formatting conference papers. Please ensure that all template text is removed from your conference paper prior to submission to the conference. Failure to remove the template text from your paper may result in your paper not being published.

\end{document}
